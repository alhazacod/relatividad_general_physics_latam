\documentclass{article}

\usepackage[most]{tcolorbox}
\usepackage{physics}
\usepackage{graphicx}
\usepackage{float}
\usepackage{amsmath}
\usepackage{amssymb}


\usepackage[utf8]{inputenc}
\usepackage[a4paper, margin=1in]{geometry} % Controla los márgenes
\usepackage{titling}

\title{Tarea 4 Relatividad General }
\author{Manuel Garcia.}
\date{\today}

\renewcommand{\maketitlehooka}{%
  \centering
  \vspace*{0.05cm} % Espacio vertical antes del título
}

\renewcommand{\maketitlehookd}{%
  \vspace*{2cm} % Espacio vertical después de la fecha
}

\newcommand{\caja}[3]{%
  \begin{tcolorbox}[colback=#1!5!white,colframe=#1!25!black,title=#2]
    #3
  \end{tcolorbox}%
}

\begin{document}
\maketitle

\section{Preambulo: Agujeros Negros}

%%%%%%%%%%%%%%%%%%%%%%%%%%%%%%%%%%%%%%%%%%%%%%%%%%%%%%%%%%%%%%%%%%%%%

\section{Agujero Negro de Reissner-Nordstrom}

\hfill

\textbf{a) }

\textbf{b) } Buscamos los valores de $ r  $ donde $ g ^ {rr } = 0  $
\begin{gather*}
  1 - \frac{2M }{r} + \frac{Q^2 }{r^2 } = 0 \\
  r^2 - 2Mr + Q^2 = 0 \\
  r _{\pm } = M \pm \sqrt{M^2 - Q^2 } 
\end{gather*}
Para $ M > \left|Q \right| $
\begin{gather*}
  r _{\pm } = M \pm \sqrt{M^2 - Q^2 }  
\end{gather*}
Para $ M = \left|Q \right| $
\begin{gather*}
  r _{+  } = r _{-   } = M 
\end{gather*}
Para $ M < \left|Q \right| $ No hay soluciones reales por lo que no existen horizontes de eventos.


\textbf{c) } 
\begin{gather*}
  v= t + r^* \quad \rightarrow \quad dt = dv + dr^* \qquad \qquad dr^* = \frac{dr }{f(r)} 
\end{gather*}

Reemplazando en la metrica 
\begin{align*}
  ds^2 &= - f(r) (dv-dr^*)^2 + f^{-1}\left(r\right) dr^2 + r^2 d\Omega \\
       &= - f(r) dv^2 + 2 f(r) dv dr^* - f(r) dr^* + f(r) dr^2 + r^2 d\Omega \\
       &= - f(r) dv^2 + 2 f(r) dv \frac{dr }{f(r) } - f(r) \frac{dr^2 }{f^2(r) } + f(r) dr^2 + r^2 d\Omega \\
       &= - f(r) dv^2 + 2 dvdr + r^2 d\Omega
\end{align*}

Para hallar la singularidad de la metrica: 
\begin{gather*}
  f(r) = 1 - \frac{2M }{r } + \frac{Q^2 }{r^2 } \quad \rightarrow \quad \infty \qquad \text{Cuando } r \rightarrow 0  
\end{gather*}
Por lo que existe una singularidad en $ r= 0 $.


\end{document}
