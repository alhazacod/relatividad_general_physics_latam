\documentclass{article}

\usepackage[most]{tcolorbox}
\usepackage{physics}
\usepackage{graphicx}
\usepackage{float}
\usepackage{amsmath}
\usepackage{amssymb}


\usepackage[utf8]{inputenc}
\usepackage[a4paper, margin=1in]{geometry} % Controla los márgenes
\usepackage{titling}

\title{Tarea \#5 Relatividad General Physics Latam }
\author{Manuel Garcia.}
\date{\today}

\renewcommand{\maketitlehooka}{%
  \centering
  \vspace*{0.05cm} % Espacio vertical antes del título
}

\renewcommand{\maketitlehookd}{%
  \vspace*{2cm} % Espacio vertical después de la fecha
}

\newcommand{\caja}[3]{%
  \begin{tcolorbox}[colback=#1!5!white,colframe=#1!25!black,title=#2]
    #3
  \end{tcolorbox}%
}

\begin{document}
\maketitle

\section{Agujeros negro en otras dimensiones }
\begin{gather}
  g^{00} = \frac{r ^ {6 }}{l ^ {6 }} \left(1 - \frac{ml^2 }{r^2 }\right) \qquad \qquad 
  g ^ {11 } = -\frac{l^2 }{r^2 }\left(1 - \frac{ml^2 }{r^2 }\right)^ {-1 } \\ 
  \frac{d g _{00 }  }{d r} = \frac{6r^5 }{l^6 } - \frac{4mr^3 }{l^4}
\end{gather}
\begin{align*}
  \kappa &= \frac{1}{2} \left[\sqrt{-g ^ {00 } g ^ {11 }}  \left|\frac{d g _{00 }  }{d r}\right| \right] _{r = r^+} \\
         &= \frac{1}{2} \left[\sqrt{\left(\frac{r^6 }{l^6 } - \frac{mr^4 }{l^4 }\right) \left(\frac{r^2 }{l^2 } - m\right)^ {-1}} \left(\frac{6r^5 }{l^6 } - \frac{4mr^3 }{l^4}\right) \right] \\
         &= \left. \frac{r^5 }{l^6 }\left[\frac{3r^2 }{l^2 } - 2m\right] \right|_{r= r^+}  \\
         &= \frac{r_+^5 }{l^6 }
\end{align*}
Reemplazando en la temperatura de Hawking $ T_H = \frac{\kappa}{2\pi } $
\begin{gather*}
  T_H = \frac{r^5 }{2\pi l^6 } 
\end{gather*}





\section{Identidades útiles }
\begin{itemize}
  \item 
    \[
    |M|^{-1} \delta |M| = M^{ik} \delta M_{ik}
    \]

    Sabemos que la variación del determinante de una matriz \( M \) se puede expresar como:

    \[
    \delta |M| = |M| \text{tr}(M^{-1} \delta M)
    \]

    Multiplicando por \( |M|^{-1} \):

    \[
    \frac{\delta |M|}{|M|} = \text{tr}(M^{-1} \delta M) = M ^ {ik } \delta M _{ik} 
    \]
  \item 
    \[
    \text{tr} \left( M - \frac{1}{D} \text{tr}(M) \right)^2 = \text{tr}(M^2) - \frac{1}{D} (\text{tr}(M))^2
    \]

    Expandiendo la traza del cuadrado:
    \begin{align*}
      \text{tr} \left( M - \frac{1}{D} \text{tr}(M) \right)^2 &= \text{tr} \left( M^2 - 2\frac{1}{D} \text{tr}(M) M + \frac{1}{D^2} (\text{tr}(M))^2 I \right)\\
              &=\text{tr}(M^2) - \frac{2}{D} \text{tr}(M) \text{tr}(M) + \frac{1}{D^2} (\text{tr}(M))^2 \text{tr}(I)  \\
              &= \text{tr}(M^2) - \frac{2}{D} (\text{tr}(M))^2 + \frac{1}{D} (\text{tr}(M))^2 \\
              &= \text{tr}(M^2) - \frac{1}{D} (\text{tr}(M))^2
    \end{align*}



  \item 
    consideremos el "shear" \(\sigma_{ij}\), definido como:

    \[
    \sigma_{ij} \equiv \frac{1}{2} \left( g^{ik} g_{jk} - \frac{1}{D-1} \delta_j^k g^{kl} g_{il} \right)
    \]

    Queremos demostrar que la expresión para \( \text{tr}(\sigma^2) \) se puede escribir como:

    \[
    \text{tr}(\sigma^2) = \frac{1}{4} \left( g^{ik} g_{jk} g^{jl} g_{il} \right) - \frac{1}{D-1} \theta^2
    \]

    Donde:

    \[
    \theta \equiv \frac{1}{2} g^{ik} \dot{g}_{ik}
    \]

    Primero, calculamos \( \text{tr}(\sigma^2) \):

    \[
    \text{tr}(\sigma^2) = \sigma_j^i \sigma_i^j
    \]

    Sustituyendo \( \sigma_{ij} \) en esta expresión:

    \[
    \sigma_j^i \sigma_i^j = \frac{1}{4} \left( g^{ik} g_{jk} - \frac{1}{D-1} \delta_j^k g^{kl} g_{il} \right) \left( g^{jl} g_{il} - \frac{1}{D-1} \delta_i^l g^{lm} g_{jm} \right)
    \]

    Expandiendo y simplificando usando las propiedades de las trazas y de los productos matriciales, llegamos a:

    \[
    \text{tr}(\sigma^2) = \frac{1}{4} \left( g^{ik} g_{jk} g^{jl} g_{il} \right) - \frac{1}{D-1} \theta^2
    \]
\end{itemize}








\section{Edad del universo }

\[
H^2(t) = \frac{8\pi G}{3} \left[ \rho(t) + \frac{\rho_r - \rho_0}{a^2(t)} \right]
\]
\[
dt = H_0^{-1} \frac{da}{a} \left[ \Omega_\Lambda + \frac{1 - \Omega_\Lambda}{a^3} \right]^{-1/2}
\]
\begin{enumerate}
  \item[\textbf{a)}] 
    Si \(\Omega_\Lambda = 0\), la ecuación se simplifica:
    \[
    dt_0 = H_0^{-1} \frac{da}{a} \left[ \frac{1}{a^3} \right]^{-1/2} = H_0^{-1} a ^ {1/2} da
  \]
    La integral es:
    \[
    t _{0}  = H_0^{-1} \int_0^1 a^{1/2} da
    \]
    \[
    t _{0}  = \left. H_0^{-1} \frac{2}{3} a ^ {3/2 } \right| _{0}^1 = \frac{2}{3 H_0} 
  \]

  \item[\textbf{b)}] 
    Para \(\Omega_\Lambda = 0.7\), la ecuación es:
    \[
    dt _{0.7}= H_0^{-1} \frac{da}{a} \left[ 0.7 + \frac{0.3}{a^3} \right]^{-1/2}
    \]
    La integral es:
    \[
      t_{0.7} = H_0^{-1} \int_0^1 \left[ \frac{0.7 }{a^2 } + \frac{0.3 }{a^5} \right]^{-1/2} da
    \]
    Esta integral se resuelve numéricamente obteniendo que 
    \begin{gather*}
      t _{0.7 }  = \frac{0.36}{H_0} 
    \end{gather*}

\end{enumerate}

\subsection*{(b) Integral para \(\Omega_\Lambda = 0.7\)}
Para \(\Omega_\Lambda = 0.7\), la ecuación es:
\[
dt _{0.7}= H_0^{-1} \frac{da}{a} \left[ 0.7 + \frac{0.3}{a^3} \right]^{-1/2}
\]
La integral es:
\[
  t_{0.7} = H_0^{-1} \int_0^1 \frac{da}{a \left[ 0.7 + \frac{0.3}{a^3} \right]^{1/2}}
\]
Esta integral se resuelve numéricamente.

\subsection*{(c) Repetición del cálculo para diferentes valores de \(h\)}
Repetimos el cálculo para \(h = 0.67\), \(h = 0.73\) y \(h = 0.69\).

Para \(\Omega_\Lambda = 0\):
\[
t_0 = \frac{2}{H_0} = \frac{2}{100 h \text{ km/s/Mpc}} = \frac{2}{100 \cdot 3.24 \times 10^{-18} h \text{ s}^{-1}}
\]

Para \(\Omega_\Lambda = 0.7\), \(t_0\) se calculará numéricamente, pero estará afectado por el valor de \(H_0\) (que depende de \(h\)).

\end{document}
