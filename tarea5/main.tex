\documentclass{article}

\usepackage[most]{tcolorbox}
\usepackage{physics}
\usepackage{graphicx}
\usepackage{float}
\usepackage{amsmath}
\usepackage{amssymb}


\usepackage[utf8]{inputenc}
\usepackage[a4paper, margin=1in]{geometry} % Controla los márgenes
\usepackage{titling}

\title{Tarea \#5 Relatividad General Physics Latam }
\author{Manuel Garcia.}
\date{\today}

\renewcommand{\maketitlehooka}{%
  \centering
  \vspace*{0.05cm} % Espacio vertical antes del título
}

\renewcommand{\maketitlehookd}{%
  \vspace*{2cm} % Espacio vertical después de la fecha
}

\newcommand{\caja}[3]{%
  \begin{tcolorbox}[colback=#1!5!white,colframe=#1!25!black,title=#2]
    #3
  \end{tcolorbox}%
}

\begin{document}
\maketitle

\section{Agujeros negro en otras dimensiones }
\begin{itemize}
  \item 
    Para $ z  $ general 
    \begin{align*}
      \kappa &= \left[\sqrt{\frac{l ^ {2z }}{r ^ {2z }} \left(1 - \frac{ml^2 }{r^2 }\right)^ {-1} \ \frac{r^2 }{l^2 } \left(1 - \frac{ml^2 }{r^2}\right)} \left(\frac{z r ^ {2z-1 }}{l ^ {2z }} - \frac{m r ^ {2z-3 }(z-2)}{l ^ {2z-2}}\right) \right] _{r = r_+ } 
      \\
             &= \frac{l ^ {z - 1 }}{r_+ ^ {z - 1 }} \left(\frac{z r_+ ^ {2z-1 }}{l ^ {2z }} - \frac{m r_+ ^ {2z-3 }(z-1)}{l ^ {2z-2}}\right) 
    \end{align*}
    \begin{gather*}
      T_H = \frac{\kappa}{2\pi} = \frac{1}{2\pi} \frac{l ^ {z - 1 }}{r_+ ^ {z - 1 }} \left(\frac{z r_+ ^ {2z-1 }}{l ^ {2z }} - \frac{m r_+ ^ {2z-3 }(z-2)}{l ^ {2z-2}}\right)
    \end{gather*}
  \item 
    Para $ z = 3  $ y utilizando que $ r_+ = \sqrt{m } l $
    \begin{gather}
      T_H = \frac{r_+^3 }{2\pi l^4 }
    \end{gather} 
  \item 
    Para demostrar que $ dM = T_{bh} dS_{bh} $ 
    \begin{gather*}
      T_{bh} = \frac{r_+^3 }{2\pi l^4 } \qquad\qquad S _{bh } = \frac{2\pi r_+ }{G_3} \quad \rightarrow \quad dS _{bh }  = \frac{2\pi}{G_3 } dr_+ 
      \\
      M = \frac{r_+^4 }{4G_3 l^4 } \quad \rightarrow \quad dM = \frac{r_+^3 }{G_3 l^4 }dr_+ \\
      dM = \left(\frac{r_+^3 }{2\pi l^4 }\right) \left(\frac{2\pi }{G_3 } dr_+\right) = T _{bh } d S _{bh } 
    \end{gather*}
\end{itemize}










\section{Identidades útiles }
\begin{itemize}
  \item 
    \[
    |M|^{-1} \delta |M| = M^{ik} \delta M_{ik}
    \]

    Sabemos que la variación del determinante de una matriz \( M \) se puede expresar como:

    \[
    \delta |M| = |M| \text{tr}(M^{-1} \delta M)
    \]

    Multiplicando por \( |M|^{-1} \):

    \[
    \frac{\delta |M|}{|M|} = \text{tr}(M^{-1} \delta M) = M ^ {ik } \delta M _{ik} 
    \]
  \item 
    \[
    \text{tr} \left( M - \frac{1}{D} \text{tr}(M) \right)^2 = \text{tr}(M^2) - \frac{1}{D} (\text{tr}(M))^2
    \]

    Expandiendo la traza del cuadrado:
    \begin{align*}
      \text{tr} \left( M - \frac{1}{D} \text{tr}(M) \right)^2 &= \text{tr} \left( M^2 - 2\frac{1}{D} \text{tr}(M) M + \frac{1}{D^2} (\text{tr}(M))^2 I \right)\\
              &=\text{tr}(M^2) - \frac{2}{D} \text{tr}(M) \text{tr}(M) + \frac{1}{D^2} (\text{tr}(M))^2 \text{tr}(I)  \\
              &= \text{tr}(M^2) - \frac{2}{D} (\text{tr}(M))^2 + \frac{1}{D} (\text{tr}(M))^2 \\
              &= \text{tr}(M^2) - \frac{1}{D} (\text{tr}(M))^2
    \end{align*}



  \item 
    consideremos el "shear" \(\sigma_{ij}\), definido como:

    \[
    \sigma_{ij} \equiv \frac{1}{2} \left( g^{ik} g_{jk} - \frac{1}{D-1} \delta_j^k g^{kl} g_{il} \right)
    \]

    Queremos demostrar que la expresión para \( \text{tr}(\sigma^2) \) se puede escribir como:

    \[
    \text{tr}(\sigma^2) = \frac{1}{4} \left( g^{ik} g_{jk} g^{jl} g_{il} \right) - \frac{1}{D-1} \theta^2
    \]

    Donde:

    \[
    \theta \equiv \frac{1}{2} g^{ik} \dot{g}_{ik}
    \]

    Primero, calculamos \( \text{tr}(\sigma^2) \):

    \[
    \text{tr}(\sigma^2) = \sigma_j^i \sigma_i^j
    \]

    Sustituyendo \( \sigma_{ij} \) en esta expresión:

    \[
    \sigma_j^i \sigma_i^j = \frac{1}{4} \left( g^{ik} g_{jk} - \frac{1}{D-1} \delta_j^k g^{kl} g_{il} \right) \left( g^{jl} g_{il} - \frac{1}{D-1} \delta_i^l g^{lm} g_{jm} \right)
    \]

    Expandiendo y simplificando usando las propiedades de las trazas y de los productos matriciales, llegamos a:

    \[
    \text{tr}(\sigma^2) = \frac{1}{4} \left( g^{ik} g_{jk} g^{jl} g_{il} \right) - \frac{1}{D-1} \theta^2
    \]
\end{itemize}








\section{Edad del universo }

\[
H^2(t) = \frac{8\pi G}{3} \left[ \rho(t) + \frac{\rho_r - \rho_0}{a^2(t)} \right]
\]
\[
dt = H_0^{-1} \frac{da}{a} \left[ \Omega_\Lambda + \frac{1 - \Omega_\Lambda}{a^3} \right]^{-1/2}
\]
\begin{enumerate}
  \item[\textbf{a)}] 
    Si \(\Omega_\Lambda = 0\), la ecuación se simplifica:
    \[
    dt_0 = H_0^{-1} \frac{da}{a} \left[ \frac{1}{a^3} \right]^{-1/2} = H_0^{-1} a ^ {1/2} da
  \]
    La integral es:
    \[
    t _{0}  = H_0^{-1} \int_0^1 a^{1/2} da
    \]
    \[
    t _{0}  = \left. H_0^{-1} \frac{2}{3} a ^ {3/2 } \right| _{0}^1 = \frac{2}{3 H_0} 
  \]

  \item[\textbf{b)}] 
    Para \(\Omega_\Lambda = 0.7\), la ecuación es:
    \[
    dt _{0.7}= H_0^{-1} \frac{da}{a} \left[ 0.7 + \frac{0.3}{a^3} \right]^{-1/2}
    \]
    La integral es:
    \[
      t_{0.7} = H_0^{-1} \int_0^1 \left[ \frac{0.7 }{a^2 } + \frac{0.3 }{a^5} \right]^{-1/2} da
    \]
    Esta integral se resuelve numéricamente obteniendo que 
    \begin{gather*}
      t _{0.7 }  = \frac{0.36}{H_0} 
    \end{gather*}
    Este universo es mas antiguo.

\end{enumerate}







\section{Integrando las ecuaciones de Boltzman }
\begin{itemize}
  \item[\textbf{a)}] 
    \begin{gather*}
      \frac{\partial n  }{\partial t } = - 3 \frac{\dot a }{a } n - n \bar n \expval{\sigma v } + P(t)
    \end{gather*}
    Termino $ -3 \frac{\dot a }{a }n  $, este termino muestra la dilución de partículas debido a la expasión. Termino $ n\bar n \expval{\sigma v } $, este termino representa la aniquilación de partículas y antipartículas; $ \bar n  $ es la densidad numérica de antipartículas; $ \sigma $ es la sección eficaz de aniquilación; y $ v  $ es la velocidad relativa entre las partículas y antipartículas; el promedio $ \expval{\sigma v }  $ es la tasa de aniquilación. El termino $ P(t)  $ representa la producción de partículas.
\end{itemize}

\end{document}
