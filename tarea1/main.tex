\documentclass{article}

\usepackage[most]{tcolorbox}
\usepackage{physics}
\usepackage{graphicx}
\usepackage{float}
\usepackage{amsmath}
\usepackage{amssymb}


\usepackage[utf8]{inputenc}
\usepackage[a4paper, margin=1in]{geometry} % Controla los márgenes
\usepackage{titling}

\title{Taller \#1 Relatividad General Physics Latam }
\author{Manuel Garcia.}
\date{\today}

\renewcommand{\maketitlehooka}{%
  \centering
  \vspace*{0.05cm} % Espacio vertical antes del título
}

\renewcommand{\maketitlehookd}{%
  \vspace*{2cm} % Espacio vertical después de la fecha
}

\newcommand{\caja}[3]{%
  \begin{tcolorbox}[colback=#1!5!white,colframe=#1!25!black,title=#2]
    #3
  \end{tcolorbox}%
}

\begin{document}
\maketitle

\section{Manifold Diferencial}
\textit{a)} 
\begin{itemize}
  \item \textbf{Espacio Topologico } Sea $ X  $ un conjunto y sea $ T  $ una colección de subconjuntos de $ X  $. Se llama espacio topologico a $ (X,T) $ si cumplen las siguientes propiedades:
    \begin{itemize}
      \item $ \emptyset $ y $ X  $ están en $ T  $.
      \item Para $ \{u_i \in T \ | \ i\in I \}, \quad \underset{i }{\bigcup}\  u_i \in T $
      \item Si $ u_1,u_2 \in T \ \rightarrow \ u_1 \cap u_2 \in T $
    \end{itemize}
    $ T  $ es llamado topologia.
  \item \textbf{Manifold Diferenciables } Sea $ M $ un espacio de Haussdorff, $ M  $ es un manifold diferenciables si tiene la siguiente estructura:
    \begin{itemize}
      \item Sea $ M = U_\alpha U_\alpha $ de un recubrimiento abierto
      \item Hay un mapa continuo e invertible $ \phi_\alpha \ : \ u_\alpha \ \rightarrow \ \phi_\alpha(u_\alpha) \subseteq \mathbb{R}^n $
      \item Para todo $ \alpha,\beta $ tenemos $ \phi_\alpha(u_\alpha\cap u_\beta) $ es abierto en $ \mathbb{R}^n  $, y las funciones de transición: 
        \begin{gather*}
          \phi_\alpha \circ \phi_\beta ^ {-1 }: \ \phi_\beta(u_\alpha \cap u_\beta) \ \rightarrow \ \phi_\alpha(u_\alpha \cap u_\beta)
        \end{gather*}
        Son $ C^\infty $-funciones. $ (u_\alpha, \phi_\alpha) $ es llamado un chart de coordenadas y $ \{(u_\alpha, \phi_\alpha)\}_\alpha $ es llamado Atlas.
    \end{itemize}
  \item \textbf{Manifold Diferenciable con Frontera} 
\end{itemize}


\end{document}
